\documentclass[pdftex,a4paper,12pt]{report}
\newtheorem{theorem}{Theorem}[section]
\newtheorem{lemma}[theorem]{Lemma}
\newtheorem{proposition}[theorem]{Proposition}
\newtheorem{corollary}[theorem]{Corollary}

\newenvironment{proof}[1][Proof]{\begin{trivlist}
\item[\hskip \labelsep {\bfseries #1}]}{\end{trivlist}}
\newenvironment{definition}[1][Definition]{\begin{trivlist}
\item[\hskip \labelsep {\bfseries #1}]}{\end{trivlist}}
\newenvironment{example}[1][Example]{\begin{trivlist}
\item[\hskip \labelsep {\bfseries #1}]}{\end{trivlist}}
\newenvironment{remark}[1][Remark]{\begin{trivlist}
\item[\hskip \labelsep {\bfseries #1}]}{\end{trivlist}}

\newcommand{\qed}{\nobreak \ifvmode \relax \else
      \ifdim\lastskip<1.5em \hskip-\lastskip
      \hskip1.5em plus0em minus0.5em \fi \nobreak
      \vrule height0.75em width0.5em depth0.25em\fi}
\def\therefore{
\leavevmode
\lower0.1ex\hbox{$\bullet$}
\kern-0.2em\raise0.7ex\hbox{$\bullet$}
\kern-0.2em\lower0.2ex\hbox{$\bullet$}
\thinspace}
\usepackage[linesnumbered,boxruled,vlined]{algorithm2e}
\usepackage{amsmath}
\usepackage{multicol}
\usepackage{hyperref}
\usepackage{color}
\usepackage{fullpage}
\usepackage{graphicx}
\newcommand{\abs}[1] {$\mid #1 \mid$}
\renewcommand{\thesection}{\arabic{section}}

\hypersetup{
  colorlinks,
  citecolor=blue,
  linkcolor=blue,
  urlcolor=blue
}
\begin{document}
\begin{titlepage}
\begin{center}

\textsc{\LARGE IIT Kanpur}\\[1.5cm]

\textsc{\Large CS345A - AlgorithmsII}\\[0.5cm]

% Title
{ \huge \bfseries Assignment 4 \\[0.4cm] }


\begin{minipage}{0.4\textwidth}
\begin{flushleft} \large
Anjani Kumar\\
11101
\end{flushleft}
\end{minipage}
\begin{minipage}{0.4\textwidth}
\begin{flushright} \large
Sumedh Masulkar\\
11736
\end{flushright}
\end{minipage}

\vfill

% Bottom of the page
{\large March 11, 2014}

\end{center}
\end{titlepage}

\tableofcontents
\newpage

\section{Shortest path: An adventurous drive in Thar desert}
\subsection{Problem}
Given a graph $G$ containing nodes \{$s,d,junctions$\}, out of which some junctions are fuel-stations, Edges: the roads connecting the nodes, and the maximum capacity $c$ of the bike before it needs refilling. We need to find shortest feasible path from $s\ to\ d$, if it exists, otherwise notify that it does not exists. 
\subsection{Brief Description:}
The problem can be modelled using 2 graphs:\\
$G(V,E)$: 
\begin{itemize}
\item The nodes V are junctions present in the desert.
\item The edges E are the roads connecting those junctions.
\end{itemize}
$G_{f}(V_{f},E_{f})$:
\begin{itemize}
\item The nodes $V_{f}$ are the start, end, and the fuel stations.
\item The edges $E_{f}$ are the \emph{feasible} roads connecting those junctions.
\item Initially, $G_{f}$ contains no edges.
\end{itemize}
The algorithm returns the shortest feasible path between $s$ and $d$, if possible. Else, it returns no path.
\\
We are using \textbf{Dijkstra's} algorithm, multiple times to find the feasible path.
\\\\
Construction of $G_{f}$:
\begin{itemize}
\item insert $s$
\item Begin Dijkstra's algorithm from $s$ in graph $G$ and insert the edges from $s$ to other nodes of $G_{f}$ accordingly.
\item For all fuel stations in $G_{f}$, begin Dijkstra's algorithm from them in graph $G$, update the edges in $G_{f}$ according to minimum distance between nodes.
\item Delete all the edges in $G_{f}$ whose weights are more than capacity of bike ($c$).
\end{itemize}


Functions used:
\begin{itemize}
\item Dijkstra's\_Mod($G_{f}$,$G$,$u$)
\\
Runs Dijkstra's algorithm from node $u$ in graph $G$ and update edges in graph $G_{f}$ accordingly.
\item update($G$,$u$,$v$,$w$)
\\
In the input graph $G$, it inserts an edge between nodes $u$ and $v$ with weight $w$, if there was no edge previously. Otherwise, if $w\ <\ weight_{(u,v)}$ it updates the weight of edge ($u$,$v$) to $w$.
\item Delete($G$,$c$)
\\
Deletes any edge in input graph $G$ whose weight is more than $c$.
\item Pathfinder($T$,$u$,$v$)
\\
Returns the path from $u$ to $v$ in tree $T$ using BFS traversal, if it exists. Else it returns NULL. 
\end{itemize}

\subsection{Pseudo Code:}
\begin{algorithm}
        Tree $\gets$ Dijkstra's($G$,$u$)\\
        \For{$\forall$ $v$ $\in$ $T$} {
            \If{$v$ $\in$ \{$s,d,fuel-pump$\}}{
                Insert($G_{f}$,$u$,$v$,$w$)
            }
        }
        \caption{Pseudo-code for Dijkstra's\_Mod($G_{f}$,$G$,$u$)}
\end{algorithm}

\begin{algorithm}
        Create a graph $G_{f}(V_{f},E_{f})$ with $V_{f}$ $\gets$ \{$s,d,fuel-stations$\}\\
        $E_{f}$ $\gets$ NULL\\
       Dijkstra's\_Mod($G_{f}$,$G$,$s$)\\
       \For{$\forall$ u $\in$ $G_{f}$ - \{$s$\}}{
        Dijkstra's\_Mod($G_{f}$,$G$,$u$)
       }
       Delete($G_{f},c$)\\
       Path $\gets$ Dijkstra's($G_{f},s$)\\
       $\Return$ Pathfinder($Path,s,d$)
    \caption{Pseudo-code for Find\_Path($G,s,d,c$)}
\end{algorithm}

        
\subsection{Proof Of Correctness}
2 cases are possible for a given input G:
\begin{itemize}
\item Case 1: If there is no feasible path in $G$.
\item Case 2: If $\exists$ a feasible path from $s$ to $d$ in $G$.
\end{itemize}
\textbf{Claim:} \emph{In case 1, there would be no path from $s$ to $d$ in graph $G_{f}$}.\\
\textbf{Proof:} Case 1 can be further divided into 2 cases:
\begin{itemize}
\item Case 1A: There is no path from $s$ to $d$ in $G$.
    \begin{itemize}
    \item We are using the result of dijkstra's algorithm applied on graph $G$ to update edges in $G_{f}$. Hence, a path from $s$ to $d$ in $G_{f}$ in this case is not possible. 
    \end{itemize}
\item Case 1B: There is a path from $s$ to $d$ but it is not feasible.
    \begin{itemize}
    \item In this case, some edges in the path have weights $>$ $c$.
    \item The Delete function used by the algorithm removes all the edges $E_{f}$ in $G_{f}$ such that $w_{G_f}$ $>$ $c$. Hence, there is no path in $G_{f}$ between $s$ and $d$.
    \end{itemize}
\end{itemize}
\textbf{Claim:} \emph{In Case 2, There will always be a path from s to d in $G_{f}$}.\\
\textbf{Proof:} Case 2 can be further divided into 2 cases:
\begin{itemize}
    \item Case 2A: The path requires no refilling.
    \begin{itemize}
    \item In this case, $\exists$ a path from $s\ to\ d$ with distance $\leq\ c$.
    \item Since all the shortest edges in $G$ with $weight\ \leq\ c$ are kept in $G_{f}$, therefore $G_{f}$ will also contain that path.
    \end{itemize}
    \item Case 2B: The path requires refilling.
    \begin{itemize}
    \item In this case, the path has distance $>\ c$, and has edges with $weight\ \leq\ c$.
    \item Using the \textbf{shortest sub-path} rule, if the given feasible path between $s$ and $d$ is the shortest, then all its edges will also have the minimum weight possible.
    \item Since we are using \textbf{Dijkstra's} algorithm to build edges in $g_{f}$, at the end of the algorithm, all the edges in $G_{f}$ will be of minimum possible weight.
    \item Hence the optimal path will also be present in $G_{f}$.
    \end{itemize}
\end{itemize}


\subsection{Time Complexity}
\begin{itemize}
\item \textbf{Update} take $O(1)$ time.
\item \textbf{Dijkstra's\_Mod} takes $O(m log\ n)$ time.
\item \textbf{Delete} takes $O(m)$ time.
\item \textbf{Pathfinder} performs simple BFS traversal, hence it takes $O(m+n)$ time.
\item \textbf{Dijkstra's\_Mod} is called fo evey node in $G_{f}$.
\item Hence the total time taken T is:
\begin{center}
$T = c_{1}O(mlog\ n) + c_{2}nO(mlog\ n) + c_{3}O(m+n)$\\
T = $O(mnlog\ n)$
\end{center}

\end{itemize}
\newpage
\section{Optimizing your time during exam days}
\subsection{Brief description}
\begin{itemize}
\item In order to maximize the average, we need to maximize the sum of all the grades obtained in different courses.\\
\item The problem can be solved using Dynamic Programming.\\
\item The optimal solution for the grades would follow the given recursion:
\begin{center}
\label{eqn}
$Opt(i,h)\ =\ max_{j\ \in\ 0\ to\ h}(f_{i}(j)\ +\ Opt(i-1,h-j))$
\end{center}
\item We are using 2 matrix: $grade$ and $max\_time$.
%\begin{itemize}
\item $grade[i][j]$ stores the optimum grade for $i$ courses that are given a total of $j$ hours.
\item $max\_time[i][j]$ stores the hours required by the $i^{th}$ course when a total of $i$ courses are given $j$ hours.
\item For a single course, $grade[1][h]$ is simply $f_1(h)$
\item In case of finding optimal solution for multiple courses, the algorithm iteratively computes the $grade$ and $max\_time$ matrix using the formula in \ref{eqn}.
\item The hours required for each course can be calculated using back tracking.
\end{itemize}
\subsection{Functions used}
\begin{itemize}
\item Optimize($n,H$)\\
It computes the matrix $grade[i][j]$ and $max\_time[i][j]$.
\item getHours($n,H$)\\
It returns the amount of hours required for each course to achieve maximum grades using backtracking. Suppose the input is ($i,h$) Then the hours for course $i-1$ will be given by max\_time[$i-1$][$h - max\_time[i][h]$].\\
\end{itemize}

\newpage
\subsection{Pseudo code}
\begin{algorithm}
        \For{$h \gets 0$ to $H$}{
            grade[$1$][$h$] $\gets$ $f_{1}(h)$\\
             max\_time[1][h] $\gets$ h\\
            %$h++$
        }
        \For{$i \gets 2$ to $n$}{
            \For{$h \gets 0$ to $H$}{
                temp\_grade $\gets$ $0$\\
                \For{$j \gets 0$ to $h$}{
                    \If{($f_{i}(j)+grade[i-1][h-j]\ \geq\ temp\_grade$)}{
                        temp\_j $\gets$ j\\
                        $temp\_grade$ $\gets$ $f_{i}(j)+grade[i-1][h-j]$
                    }
                  %  $j++$
                }
                grade[i][h] $\gets$ temp\_grade\\
                max\_time[i][h] $\gets$ temp\_j\\
                %$h++$
            }
            %$i++$
        }
    \caption{Pseudo code for Optimize($n,H$)}
\end{algorithm}

\begin{algorithm}
        $h$ $\gets$ $H$\\
        \For{$i \gets n$ to $1$}{
            $hour[i]$ $\gets$ max\_time[$i$][$h$]\\
            $h$ $\gets$ $h$ - hour[i]\\
            %$i--$
        }
    \caption{Pseudo code for getHours($n,H$)}
\end{algorithm}

\begin{algorithm}
        Optimize($n,H$)\\
        getHours($n,H$)
    \caption{Pseudo code for main()}
\end{algorithm}
\subsection{Proof of Correctness}
To Prove:\\
\begin{center}
$Opt(i,h)\ =\ max_{j\ \in\ 0\ to\ h}(f_{i}(j)\ +\ Opt(i-1,h-j))$
\end{center}
Where $Opt(i,h)\ is\ the\ optimal\ solution\ for\ i\ courses\ given\ a\ total\ of\ h\ hours$.\\
\newpage
\textbf{Proof by Induction}\\
\begin{itemize}
\item Base Case:\\
In case of only 1 course, entire hours will be given to that course since $f_{i}(h)$ is non decreasing.
\item Induction Hypothesis:\\
Let us assume that the theorem is true for $i-1$ courses. Hence, for $\forall\ h\ \in H$, we know the optimal solution for $i-1$ courses.
\item Induction step:\\
For the $i^{th}$ course, the \textbf{Optimize} function takes the following sum for $\forall\ j\ \in h$:\\
\begin{center}
$f_{i}(j)\ +\ Opt(i-1,h-j)$
\end{center}
and keeps the maximum value among those $j$.\\
This step is repeated for $\forall\ h\ \in H$, so using optimal solution for $(i-1)^{th}$ step, we have got optimal solution for $i^{th}$ step.
\item Hence proved.
\end{itemize}
\subsection{Time Complexity}
As can be seen from the pseudo code, it take $O(nH^2)$ to perform \textbf{Optimize} and it takes $O(n)$ time to perform \textbf{getHours}.\\
Hence the overall time complexity is \textbf{$O(nH^2)$}.
\subsection{Space Complexity}
We are using 2 matrix of size $n$ x $H$. Therefore total space taken is $O(nH)$.
\section{Dynamic Programming again}
\subsection{problem}
Given a set of graphs \{$G_0,G_1,...,G_b$\}, each containing $n$ nodes such that \{$s,t$\} $\in$ the vertices of all graphs.\\
We have to find a polynomial time algorithm to find the sequence of paths $P_0,P_1,...,P_b$ of minimum cost given by:
\begin{center}
$cost(P_0,P_1,...,P_b)\ =\  \sum_{i=0}^{b}l(P_i)\ +\ k*changes(P_0,P_1,...,P_b) $
\end{center}
where, $l(P_i)$ is the number of edges in $P_i$\\
and $changes(P_0,P_1,...,P_b)$ is the number of indices $i$ $(0\leq\ i\ \leq\ b-1)$ for which $P_i \neq P_{i+1}$ 
\subsection{Brief Description}
\begin{itemize}
\item The problem is solved using Dynamic programming.
\item Let G = \{$G_0,G_1,...,G_b$\} be a set of graphs with $n$ nodes in each graph.
\item We define Intersect($i,j$) as:
\begin{center}
$Intersect(i,j)\ =\ \cap_{k=i}^{k=j}(G_{k}) \forall (0 \leq i \leq j \leq b)$
\end{center}
The graph $Intersect(i,j)$ obtained contains n nodes and the common edges in graphs $G_i,G_{i+1}...,G_j$. Each $Intersect(i,j)$ is given the following parameters:
\begin{itemize}
\item $shortest(i,j)$: It is the number of edges in the shortest path from $s$ to $t$ in $Intersect(i,j)$.
\item $sPath(i,j)$: It is the shortest path from $s$ to $t$ in $Intersect(i,j)$.
\end{itemize}
\item Define $costMin(i)$ for a graph $G_i$ such that $costMin(i)$ stores the minimum of $cost(P_0,P_1,...,P_i)$ in the set of graphs \{$G_0,G_1,...,G_i$\}. The following recursion holds for $costMin(i)$:\\
$costMin(i)$=
\begin{center}
$ min((i+1)*shortest(0,i),min_{j=0}^{j=i}(costMin(j)+(i-j)*shortest(j+1,i)+k))$
\end{center}
\item This computation can be done iteratively by incrementing i from ($0\ to\ b$) as $costMin(i)$ depends on $shortest(i,j)$ and \{$costMin(j)\ \forall\ 0\leq j \leq i$\}; both of which have already been computed beforehand.
\item The algorithm maintains the sequence $(P_0,P_1,...)$ in the following manner:
\begin{itemize}
\item Suppose the $(i-1)^{th}$ sequence $(P_0,P_1,...,P_{i-1})$ is computed correctly.
\item In the $i^{th}$ iteration, $costMin(i)$ comes out to be minimum for some  ($0\leq j \leq i$). In that case, the sequence $(P_{j+1},...,P_i)$ is updated to $sPath(j,i)$ and the rest of the sequence remains as it is.
\end{itemize}
\end{itemize}
\subsection{Functions used}
\begin{itemize}
\item $pathfinder(G,s,t)$: It performs BFS traversal on graph \\
$Intersect(i,j)$ = $\cap_{k=i}^{k=j}(G_{k}) \forall (0 \leq i \leq j \leq b)$ and returns the shortest path, and the length of that path in $Intersect(i,j)$.
\end{itemize}
\newpage
\subsection{Pseudo code}
\begin{algorithm}
$pathfinder(G,shortest,sPath)$
\For{$i \gets 0$ to $b$}{
            $min \gets (i+1)*shortest(0,i)$\\
            $pos \gets 0$\\
            \For{$j \gets 0$ to $i-1$}{
                \If{$min > costMin(j)+(i-j)*shortest(j+1,i)+k$}{
                $min \gets costMin(j)+(i-j)*shortest(j+1,i)+k$\\
                $pos \gets j+1$\\
                }
            }
            $costMin(i) \gets min$\\
            \For{$j \gets 0$ to $i$}{
            \If{$j \leq pos-1$}{
            $minPath[i][j] \gets minPath[pos-1][j]$\\
            \Else{
            $minPath[i][j] \gets sPath(pos,i)$\\
            }
            }
            }
        }
\Return{minPath[b]} 
\caption{Pseudo code for sequence(G,s,t,b,k)}
\end{algorithm}


\subsection{Proof of Correctness}
To Prove:\\
$costMin(i)$ = 
\begin{center}
$min((i+1)*shortest(0,i),min_{j=0}^{j=i}(costMin(j)+(i-j)*shortest(j+1,i)+k))$
\end{center}
Where $costMin(i)$ for a graph $G_i$ such that $costMin(i)$ stores the minimum of $cost(P_0,P_1,...,P_i)$ in the set of graphs \{$G_0,G_1,...,G_i$\}.\\
\textbf{Proof by Induction}\\
\begin{itemize}
\item Base Case:\\
In case of only 1 graph, shortest(0,0) will store the length of shortest path between s and t in G(0,0) , which is trivial to prove.
\item Induction Hypothesis:\\
Let us assume that the theorem is true for $i-1$ graphs.
\item Induction step:\\
After inserting $i^{th}$ graph, $\exists$ an optimal path sequences $(P_0,P_1,...P_i)$, such that $P_i = P_{i-1} = P_{i-2}......= P_{j+1} \neq P_{j}$, the cost for graphs $(G_0,G_1,....G_i)$ will be costMin(j) + (i-j)*shortest(j+1,i) + K(penalty for $P_{j+1} \neq P_{j}$). Therefore, finding all such possibilities $\forall j < i$, and taking minimum cost will give the optimal solution for $i$.
\item Hence proved.
\end{itemize}
\subsection{Time complexity}
\begin{itemize}
\item $pathfinder(G,s,t)$: pathfinder computes $Intersect(i,j)$ $\forall (0 \leq i \leq j \leq b)$. $\exists\ O(b^2)$ pairs of $(i,j)$. For finding the intersection, atmost $b$ graphs are examined and each graph can have atmost $O(n^2)$ edges. Therefore each $(i,j)$ pair takes $O(bn^2)$ time, and for $O(b^2)$ pairs, time complexity would be $O(n^2b^3)$.
\item $sequence(G,s,t,b,k)$: From the pseudo code, its time complexity is $T(pathfinder)+C*O(b^2)$. Hence the overall time complexity is $O(n^2b^3)$.
\end{itemize}
\end{document}