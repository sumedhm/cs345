\documentclass[pdftex,a4paper,12pt]{report}

\usepackage{algorithm2e}

\begin{document}

\begin{titlepage}
\begin{center}

\textsc{\LARGE IIT Kanpur}\\[1.5cm]

\textsc{\Large CS345A - AlgorithmsII}\\[0.5cm]

% Title
{ \huge \bfseries Assignment 1 \\[0.4cm] }


\begin{minipage}{0.4\textwidth}
\begin{flushleft} \large
Anjani Kumar\\
11101
\end{flushleft}
\end{minipage}
\begin{minipage}{0.4\textwidth}
\begin{flushright} \large
Sumedh Masulkar\\
11736
\end{flushright}
\end{minipage}

\vfill

% Bottom of the page
{\large January 12, 2014}

\end{center}
\end{titlepage}

\section {Non Dominated Points}
\subsection {$\Theta$(n log n) algorithm for non-dominated points in a plane.}

\paragraph{Overview of algorithm} 
Given set of points $P$ in a plane. \\
Divide Step:
\begin{enumerate}
\item
If there is only one point in a plane, the point itself is non-dominated set of points. Hence, return $P$.
\item
If there are two points, if any point has both x and y co-ordinates both greater than that of other point, then return that point, else return $P$.
\item
Else find the x-median of the points and divide the plane into left half plane and right half plane using the median. And now call the function for both the half planes.\\
\end{enumerate}
Conquer Step:\\
Goal: Given the non-dominated points of the two half planes, merge the solution of smaller parts to get the solution of the bigger plane.\\\\
Assuming we have two sets of points $P_1$ and $P_2$, where $P_1$ is the set of non-dominated points of the left plane, and $P_2$ is the set of non-dominated points of the right plane respectively. \\\\
The x-coordinate of all the points in right plane are obviously greater than x-coordinate of all the points in the left plane. Thus, we only need to eliminate points from $P_1$ that are dominated by points in $P_2$.\\
Since x-coordinate of points in $P_2$ is always greater, we only need to look for the y-coordinates.\\
Let y be the point in $P_2$ with maximum y-coordinate. Then the dominated points in $P_1$ are all the points whose y-coordinates are less than y-coordinate of y.\\
Thus the solution of the plane will be \{ $P_1$ - \{points in $P_1$ with y-coordinate $<$ y \} \} $\cup P_2$. \\

\pagebreak

\paragraph{Pseudo-Code.} \mbox{} \\\\
\begin{algorithm}

NonDominatedPts(set of points P)\{\\
\makebox[40pt]{}//Returns set of non dominated points from $P$.

      \uIf{$\mid$P$\mid$==1} {return P\;}
      \uElseIf{$\mid$P$\mid$==2}{
	    let $p_1$ and $p_2$ be the two points in P\;
	    \uIf {$x_1 > x_2$ and $y_1 > y_2$}{ return \{$p_1$\};	\qquad	//$p_1$=($x_1$,$y_1$), and $p_2$=($x_2$,$y_2$)}
	    \uElseIf{$x_1 < x_2$ and $y_1 < y_2$} {return \{$p_2$\}\;}   
	    \uElse{return P;}
      }
      \uElse{
	    p* $\gets$ x-median(P)\;
	    (L,R) $\gets$ split(P, p*)\;
	    $P_1$ $\gets$ NonDominatedPts(L)\;
	    $P_2$ $\gets$ NonDominatedPts(R)\;
	    $P_1$ $\gets$ $P_1$ sorted along y-axis\;
	    y $\gets$ max y-coordinate in points of $P_2$\;
	    $P_1$ $\gets$ $P_1$ - \{all points in $P_1$ whose y-coordinate $\leq$ y\} \;
	    return ($P_1\cup P_2$) \;
      }
\}
\caption{O(nlogn) algorithm to find Non Dominated Points}
\end{algorithm}
\newpage

\subsection {$\Theta$(n log h) algorithm for non-dominated points in a plane.}
\paragraph{Explanation}

\newpage

\paragraph{Pseudo-Code.} \mbox{} \\\\
\begin{algorithm}

NonDominatedPts(set of points P)\{\\
\makebox[40pt]{}//Returns set of non dominated points from $P$.

      \uIf{$\mid$P$\mid$==1} {return P\;}
      \uElseIf{$\mid$P$\mid$==2}{
	    let $p_1$ and $p_2$ be the two points in P\;
	    \uIf {$x_1 > x_2$ and $y_1 > y_2$}{ return \{$p_1$\};	\qquad	//$p_1$=($x_1$,$y_1$), and $p_2$=($x_2$,$y_2$)}
	    \uElseIf{$x_1 < x_2$ and $y_1 < y_2$} {return \{$p_2$\}\;}   
	    \uElse{return P;}
      }
      \uElse{
	    p* $\gets$ x-median(P)\;
	    (L,R) $\gets$ split(P, p*)\;
	    $P_1$ $\gets$ NonDominatedPts(L)\;
	    $P_2$ $\gets$ NonDominatedPts(R)\;
	    $P_1$ $\gets$ $P_1$ sorted along y-axis\;
	    y $\gets$ max y-coordinate in points of $P_2$\;
	    $P_1$ $\gets$ $P_1$ - \{all points in $P_1$ whose y-coordinate $\leq$ y\} \;
	    return ($P_1\cup P_2$) \;
      }
\}
\caption{O(nlogh) algorithm to find Non Dominated Points}
\end{algorithm}
\newpage

\end{document}